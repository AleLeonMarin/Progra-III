\begin{center}
	\section{Web Service}

	Un servicio web es una tecnología que utiliza protocolos y estándares para intercambiar datos
	entre aplicaciones desarrolladas en diferentes lenguajes de programación y plataformas, promoviendo la interoperabilidad.

	\subsection{SOAP}
	\begin{itemize}
		\item SOAP es un protocolo que utiliza XML para el intercambio de información.
		      Define cómo los servicios web se comunican independientemente de la plataforma o el lenguaje.
		\item SOAP permite la comunicación entre empresas y clientes mediante el uso de HTTP y XML para las solicitudes y respuestas.
		\item WSDL (Web Service Definition Language) se utiliza para describir los servicios disponibles en SOAP y cómo usarlos.
	\end{itemize}

	\subsection{REST}
	\begin{itemize}
		\item REST es un estilo de arquitectura que usa HTTP para el intercambio de información, simplificando las interacciones entre cliente y servidor.
		\item n REST, las URIs identifican los recursos y los verbos HTTP (GET, POST, PUT, DELETE) determinan las operaciones a realizar.
	\end{itemize}

	SOAP es más estructurado y permite un mayor control sobre la seguridad y transacciones, mientras que REST es más ligero y eficiente para aplicaciones web.

	\subsection{JSON}
	JSON (JavaScript Object Notation) es un formato de texto ligero para almacenar e intercambiar datos. Es fácil de leer
	y escribir para los humanos y simple para las máquinas de procesar.

	\subsubsection{Caracteristicas}
	\begin{itemize}
		\item JSON usa comillas dobles para las claves y los valores, y su formato es similar a los objetos de JavaScript.
		\item Es más liviano y rápido de procesar que XML, lo que lo convierte en una alternativa popular para la transmisión de datos.
	\end{itemize}

	\subsubsection{Tipos de valores}
	\begin{itemize}
		\item Números: enteros o decimales.
		\item String: Una secuencia de caracteres Unicode.
		\item Booleanos: true o false.
		\item Arreglos: Una colección ordenada de valores.
		\item Objetos: Un conjunto de pares clave-valor..
		\item Nulo: valor nulo.
	\end{itemize}

	\subsubsection{Ventajas}
	Es fácil de entender, rápido y más ligero que XML, además de ser nativamente soportado por JavaScript.

	\subsection{HTTP}

	HTTP (Hypertext Transfer Protocol) es un protocolo de comunicación utilizado en
	la web para transferir datos entre un cliente (generalmente un navegador web) y un servidor.

	\subsubsection{Funcionamiento}

	HTTP funciona mediante el intercambio de mensajes: el cliente envía solicitudes (requests) y el servidor
	devuelve respuestas (responses). Las solicitudes utilizan métodos (verbos) como GET, POST, PUT, DELETE, etc.

	\subsubsection{Versiones}
	\begin{itemize}
		\item HTTP/1.1: Introdujo el uso de verbos como GET y POST.
		\item HTTP/2: Mejora el rendimiento mediante la compresión de cabeceras y el uso de una única conexión para múltiples solicitudes.
		\item HTTP/3: Introduce el uso de QUIC (un protocolo basado en UDP) para mejorar la velocidad y confiabilidad.
	\end{itemize}

	\subsubsection{Codigos de respuesta}
	\begin{itemize}
		\item 1xx: Respuestas informativas.
		      \begin{itemize}
			      \item 100: Continuar.
			      \item 101: Cambio de protocolo.
			      \item 102: Procesando.
		      \end{itemize}
		\item 2xx: Respuestas satisfactorias.
		      \begin{itemize}
			      \item 200: OK.
			      \item 201: Creado.
			      \item 202: Aceptado.
			      \item 203: Información no autoritativa.
			      \item 204: Sin contenido.
			      \item 205: Restablecer contenido.
			      \item 206: Contenido parcial.
		      \end{itemize}
		\item 3xx: Redirecciones.
		      \begin{itemize}
			      \item 300: Múltiples opciones.
			      \item 301: Movido permanentemente.
			      \item 302: Encontrado.
		      \end{itemize}
		\item 4xx: Errores del cliente.
		      \begin{itemize}
			      \item 400: Solicitud incorrecta.
			      \item 401: No autorizado.
			      \item 402: Pago requerido.
			      \item 403: Prohibido.
			      \item 404: No encontrado.
			      \item 405: Método no permitido.
			      \item 406: No aceptable.
			      \item 408: Tiempo de espera de solicitud.
		      \end{itemize}
		\item 5xx: Errores del servidor.
		\begin{itemize}
            \item 500: Error interno del servidor.
            \item 501: No implementado.
            \item 502: Puerta de enlace incorrecta.
            \item 503: Servicio no disponible.
            \item 504: Tiempo de espera de puerta de enlace.
            \item 505: Versión HTTP no compatible.
        \end{itemize}
	\end{itemize}

    \subsubsection{Cabeceras}
    Las cabeceras (headers) son metadatos que se envían junto con las solicitudes y respuestas HTTP.
    \subsubsection{Parametros}
    Se pueden agregar dos tipos de parámetros de PATH y de QUERY, la principal
    diferencia es que los de PATH son obligatorios ya forman parte del URL en si,
    mientras que los de QUERY son opcionales.

    \subsection{JWT}
    Un JWT es un estándar para transmitir de manera segura información entre dos partes como un usuario y 
    un servidor. Contiene un conjunto de "claims" que son declaraciones sobre el usuario.
    \subsubsection{Estructura}
    Un JWT consta de tres partes separadas por puntos:

    \begin{itemize}
        \item Header: Contiene el tipo de token y el algoritmo de encriptación.
        \item Payload: Incluye los datos o claims, como la identidad del usuario.
        \item Signature: La firma se genera usando el header, payload y una clave secreta, 
        y se utiliza para verificar la autenticidad del token.
    \end{itemize}

    \subsubsection{Firma JWT}
        La firma asegura que el token no haya sido alterado. Si alguien modifica el token, la firma no coincidirá y se podrá rechazar.
    \subsubsection{Algortimos de encriptación}
    \begin{itemize}
        \item HS256: Utiliza una clave secreta compartida entre el emisor y el receptor.
        \item RS256: Utiliza un par de claves pública y privada para la firma y validación.
    \end{itemize}


\end{center}