\begin{center}
    \section{ Bases de datos }

    Un sistema de bases de datos es una colección de datos interrelacionados 
    y un conjunto de programas que permiten a los usuarios acceder y modificar 
    estos datos. Su propósito es ofrecer una visión abstracta de los datos, 
    ocultando los detalles de cómo se almacenan.

    \subsection{Modelos de bases de datos}
    Los modelos de datos describen la estructura de las bases de datos, sus relaciones,
    y restricciones de consistencia en los niveles físico, lógico y de vistas.

    \subsection{Base de datos relacional}
    Se basan en tablas para representar datos y relaciones entre ellos. 
    El lenguaje SQL se utiliza comúnmente para manipular estas bases de datos.

    \subsection{Restricciones de Integridad}

    Las restricciones de integridad aseguran que las modificaciones a la base de datos 
    no comprometan su consistencia. Esto incluye la integridad referencial, que garantiza que 
    las claves externas se correspondan con valores válidos en otras tablas.

    \subsection{SQL}
    SQL permite definir, manipular y consultar bases de datos. Las principales operaciones incluyen 
    la creación de tablas, la inserción de datos, la consulta y la actualización de registros.

    \subsection{Normalización}
    La normalización es un proceso para estructurar las bases de datos, eliminando redundancias y 
    asegurando la consistencia de los datos. Se utilizan formas normales para guiar este proceso, 
    ayudando a organizar los datos de manera eficiente.

    \subsubsection{Reglas de normalización}

        \begin{itemize}
            \item \textbf{Primera forma normal (1FN):} Elimina la repetición de datos. Cada valor en una 
            columna debe ser atómico, es decir, no debe contener múltiples valores. 
            Ejemplo: en lugar de repetir los datos de un cliente en cada venta, 
            se crea una tabla separada de clientes, referenciada por una clave.
            \item \textbf{Segunda forma normal (2FN):} Asegura que todas las columnas de una tabla dependan 
            únicamente de la clave primaria. Si algunas columnas no dependen de la clave principal, 
            deben separarse en nuevas tablas. Ejemplo: crear una tabla separada para detalles repetidos, 
            como fechas de ventas o productos vendidos.
            \item \textbf{Tercera forma normal (3FN):} Elimina dependencias transitivas, donde una columna no clave 
            depende de otra columna no clave. Ejemplo: los detalles del proveedor de un producto deben estar 
            en una tabla separada si no dependen de la clave principal de la tabla de ventas.
        \end{itemize}

        La normalización mejora el diseño de las bases de datos al reducir la redundancia y 
        mejorar la consistencia de los datos, haciendo que las bases sean más fáciles de mantener 
        y ampliar.


\end{center}